\section{D-modules}
    \subsection{D-modules over smooth algebraic varieties}
        \subsubsection{Finite-order differential operators}
            \begin{convention}
                Aside from rings of differential operators, most rings that will be of concern to us shall be commutative and unital.
            \end{convention}
            \begin{definition}[Differential operators] \label{def: differential_operators}
                Suppose that $\varphi: R \to S$ is a ring map and that $M, N$ are $S$-modules; in addition, fix a natural number $k \in \N$. An \textbf{differential operator} of order (at most) $k$ from $M$ to $N$ over $R$ is then an $R$-linear map $D: M \to N$ such that for all $f \in R$, the commutator $[D, f] := D(f \cdot (-)) - f \cdot D(-)$ is a differential operator of order (at most) $k - 1$ from $M$ to $N$ over $R$. 
            \end{definition}
            \begin{proposition}[Differential operators act as zero on constants] \label{prop: differential_operators_act_as_zero_on_constants}
                Suppose $\varphi: R \to S$ i a ring map, that $D: M \to N$ is a finite-order differential operator from an $S$-module $M$ to another $S$-module $N$ over $R$, and that $a \in R$ is some arbitrary element. Then $[D, \varphi(a)] = 0$.
            \end{proposition}
                \begin{proof}
                    Because the commutator map $[D, -]: S \to \Hom_R(M, N)$ is $R$-linear as a result of $D: M \to N$ being $R$-linear by definition, we have the following for all $a \in R$ and all $v \in M$, which proves the proposition:
                        $$
                            \begin{aligned}
                                [D, \varphi(a)](v) & = D(\varphi(a) \cdot v) - \varphi(a) \cdot D(v)
                                \\
                                & = \varphi(a) \cdot D(v) - \varphi(a) \cdot D(v)
                                \\
                                & = 0
                            \end{aligned}
                        $$
                \end{proof}
            \begin{proposition}[Linear combinations of differential operators] \label{prop: linear_combinations_of_differential_operators}
                Suppose that $\varphi:  R \to S$ is a ring map and that $M, N$ are $S$-modules. Then, differential operators from $M$ to $N$ over $R$ form an $S$-module $\Diff_{S/R}(M, N)$.
            \end{proposition}
                \begin{proof}
                    Observe that we have the following for all $\alpha, \alpha' \in S$, all $f \in S$, and all $v \in M$, as well as any pair of differential operators $D, D' \in \Hom_R(M, N)$:
                        $$
                            \begin{aligned}
                                    [\alpha D + \alpha' D', f](v) & = (\alpha D(f \cdot v) + \alpha' D'(f \cdot v)) - f \cdot (\alpha D(v) + \alpha' D'(v))
                                    \\
                                    & = \alpha [D, f](v) + \alpha' [D', f](v)
                                \end{aligned}
                            $$
                    From this, one infers that not only is $\alpha D + \alpha D'$ a differential operator from $M$ to $N$ over $R$, but also that differential operators from $M$ to $N$ over $R$ form an $S$-module.
                \end{proof}
            \begin{convention}[Orders of differential operators] \label{conv: orders_of_differential operators}
                \noindent
                \begin{itemize}
                    \item Suppose that $\varphi:  R \to S$ is a ring map and that $M, N$ are $S$-modules, and suppose that $D, D' \in \Hom_R(M, N)$ are two differential operators from $M$ to $N$ over $R$. We then declare that $\ord(D + D') = \max\{\ord(D), \ord(D')\}$
                    \item Now, suppose that $\varphi:  R \to S$ is a ring map and that $M, M', M''$ are $S$-modules. Suppose, also, that there are composable $R$-linear differential operators $D: M \to M'$ and $D': M' \to M''$. Then, $\ord(D' \circ D) = \ord(D') + \ord(D)$.
                \end{itemize}
            \end{convention}
            \begin{proposition}[Filtering differential operators by orders] \label{prop: filtering_differential_operators_by_orders}
                Suppose that $\varphi:  R \to S$ is a ring map and that $M, N$ are $S$-modules. Then the $S$-module $\Diff_{S/R}(M, N)$ of differential operators from $M$ to $N$ over $R$ is exhaustive\footnote{I.e. $\Diff_{S/R}(M, N) \cong \underset{k \in \N}{\colim} \Diff_{S/R}^k(M, N)$.} with respect to an increasing filtration:
                    $$S \cong \Diff_{S/R}^0(M, N) \subset \Diff_{S/R}^1(M, N) \subset ... \subset \Diff_{S/R}(M, N)$$
                wherein for all $k \in \N$, the components $\Diff_{S/R}^k(M, N)$ are $S$-submodules of $\Diff_{S/R}(M, N)$ spanned by differential operators from $M$ to $N$ of order equal to $k$. 
            \end{proposition}
                \begin{proof}
                    Clear from proposition \ref{prop: linear_combinations_of_differential_operators} and convention \ref{conv: orders_of_differential operators}.
                \end{proof}
            \begin{corollary}[Rings of differential operators] \label{coro: rings_of_differential_operators}
                For all ring maps $\varphi: R \to S$ and all $S$-modules $M$, the triple $(\Diff_{S/R}(M, M), +, \circ)$ is an $\N$-graded associative and unital $S$-algebra via the increasing filtration of $S$-modules:
                    $$S \cong \Diff_{S/R}^0(M, M) \subset \Diff_{S/R}^1(M, M) \subset ... \subset \Diff_{S/R}(M, M)$$
                Also, observe that $\Diff_{S/R}^0(M, M) \cong S$.
            \end{corollary}
            \begin{convention}[Shorthands for rings of differential operators] \label{conv: shorthands_for_rings_of_differential_operators}
                For all ring maps $\varphi: R \to S$ and all $S$-modules $M$, we shall abbreviate $\Diff_{S/R}(M, M)$ (respectively, $\Diff_{S/R}^k(M, M)$ for all $k \in \N$) by $\Diff_{S/R}(M)$ (respectively, $\Diff_{S/R}^k(M)$). If we wish to speak of $R$-linear differential operators on $S$, then we shall simply write $\D_{S/R}$ (and respectively, $\D_{S/R}^k$).
            \end{convention}
            \begin{proposition}[Principal parts] \label{prop: principal_parts}
                For all ring maps $\varphi: R \to S$ and all $S$-modules $M$, the assignment:
                    $$\Diff_{S/R}^k(M, -): S\mod \to S\mod$$
                    $$N \mapsto \Diff_{S/R}^k(M, N)$$
                is not only functorial for every $k \in \N$, but represented by an $S$-module $P_{S/R}^k(M)$, called the principal part of order $k$ of the $S$-module $M$.
            \end{proposition}
                \begin{proof}
                    
                \end{proof}
            \begin{corollary}
                For all ring maps $\varphi: R \to S$ and all $S$-modules $M$, the functor $\Diff_{S/R}^k(M, -): S\mod \to S\mod$ is left-exact. 
            \end{corollary}
            \begin{lemma}[A canonical short exact sequence] \label{lemma: differential_forms_are_differential_operators}
                For all ring maps $\varphi: R \to S$ and all $S$-modules $M$, there is a canonical short exact sequence:
                    $$0 \to \Omega_{S/R}^1 \tensor_S M \to P^1_{S/R}(M) \to M \to 0$$
                that is functorial in $M$.
            \end{lemma}
                \begin{proof}
                    
                \end{proof}
            \begin{proposition}[Differential forms are differential operators] \label{prop: differential_forms_are_differential_operators}
                Let $\varphi: R \to S$ be a ring map. Then, for all natural numbers $k \in \N$, the differential $d: \Omega_{S/R}^k \to \Omega_{S/R}^{k + 1}$ is an $R$-linear differential operator of order $1$.
            \end{proposition}
                \begin{proof}
                    
                \end{proof}
            
            \begin{lemma}[Rings of differential operators are almost commutative] \label{lemma: rings_of_differential_operators_are_almost_commutative}
                For all ring maps $\varphi: R \to S$ and all $S$-modules $M$, consider the following increasing filtration on $\Diff_{S/R}(M)$:
                    $$S \cong \Diff_{S/R}^0(M) \subset \Diff_{S/R}^{\leq 1}(M) \subset ... \subset \Diff_{S/R}^{\leq k}(M) \subset ... \subset \Diff_{S/R}(M)$$
                wherein $\Diff_{S/R}^{\leq k}(M) := \bigoplus_{0 \leq i \leq k} \Diff_{S/R}^i(M)$. With respect to this filtration, the associated graded $S$-algebra $\gr_{\bullet} \Diff_{S/R}(M)$ is commutative (and in that sense, one says that $\Diff_{S/R}(M)$ is \textbf{almost commutative}).
            \end{lemma}
                \begin{proof}
                    
                \end{proof}
            
            \begin{definition}[Weyl algebras] \label{def: weyl_algebras}
                Suppose that $R$ is a ring\footnote{Which needs not be commutative but often will be.} and that $k$ is a natural number. The \textbf{$k^{th}$ Weyl algebra} over $R$ is then the associative and unital but noncommutative algebra\footnote{Even though the name \say{Weyl} starts with the letter \say{W}, the $k^{th}$ Weyl algebra over $R$ is denoted by $A_k(R)$ because the notation $W_k(R)$ might be confused for the ring of $k$-truncated Witt vectors over $R$, especially when $R$ is a perfect ring of some prime characteristic $p$.}:
                    $$A_k(R) := \frac{R\<x_1, ..., x_k, \del_1, ..., \del_k\>}{\<\forall 1 \leq i, j \leq k: [x_i, x_j] = 0, [\del_i, \del_j] = 0, [x_i, \del_j] = \delta_{ij}\>}$$
                We can also define the \textbf{big Weyl algebra} over $R$ to be the countable pushout of $R$-algebras:
                    $$A(R) := A_1(R) \tensor_R A_2(R) \tensor_R ...$$
            \end{definition}
            \begin{lemma}[Differential operators on finite-type algebras] \label{lemma: differential_operators_on_finite_type_algebras}
                Suppose that $\varphi: R \to S$ is a ring map of finite type (say, generated by $n$ generators $x_1, ..., x_n$). Then, any $R$-linear differential operator $D: S \to S$ is of order at most $n$; in fact, there is a canonical isomorphism $\D_{S/R} \cong \D_{S/R}^n$ of $S$-algebras.
            \end{lemma}
                \begin{proof}
                    
                \end{proof}
            \begin{lemma}[Differential operators on infinite-type algebras] \label{lemma: differential_operators_on_infinite_type_algebras}
                Suppose that $R$ is a commutative ring and that $S$ is a commutative $R$-algebra that is generated as an $R$-algebra by a countable set of generators $\{x_k\}_{k \in \N}$. Then the canonical filtration $\D_{S/R}^0 \subset \D_{S/R}^1 \subset ... \subset \D_{S/R}$ does not terminate after finitely many terms.
            \end{lemma}
                \begin{proof}
                    
                \end{proof}
            \begin{theorem}[Differential operators as elements of Weyl algebras] \label{theorem: differential_operators_as_elements_of_weyl_algebras}
                Let $R$ be a commutative ring of characteristic $0$. 
                    \begin{enumerate}
                        \item For any natural number $k$, there is a canonical isomorphism $A_k(R) \cong \D_{\A^k_R}$ of $k[x_1, ..., x_n]$-algebras.
                        \item There is a canonical isomorphism $A(R) \cong \D_{\A^{\infty}_R}$ of $k[x_1, x_2, ...]$-algebras.
                    \end{enumerate}
            \end{theorem}
                \begin{proof}
                    \noindent
                    \begin{enumerate}
                        \item 
                        \item 
                    \end{enumerate}
                \end{proof}
            \begin{corollary}
                D-modules on schemes of characteristic $0$ can be be defined internally as modules over sheaves of Weyl algebras. 
            \end{corollary}
                
            \begin{definition}[D-modules] \label{def: D_modules}
                Suppose that $X$ is a ringed space over a scheme $S$ and denote the structural morphism by $\pi: X \to S$. The category of \textbf{left-D-modules} on $X$ over $S$, denoted by ${}^l\Dmod(X/S)$, or perhaps simply ${}^l\Dmod(X)$, is then full subcategory of $\QCoh(X/S)$ spanned by quasi-coherent $\calO_X$-modules upon which $\D_{X/S}$ acts on the left. The category of right-D-modules on $X$ is defined completely analogously and denoted by $\Dmod^r(X)$. 
            \end{definition}
            \begin{remark}
                
            \end{remark}
            \begin{proposition}[Base-changing differential operators] \label{prop: base_changing_differential_operators}
                Consider a pushout square as follows:
                    $$
                        \begin{tikzcd}
                        	{S'} & {R'} \\
                        	S & R
                        	\arrow[from=2-2, to=2-1]
                        	\arrow[from=2-1, to=1-1]
                        	\arrow[from=2-2, to=1-2]
                        	\arrow[from=1-2, to=1-1]
                        	\arrow["\lrcorner"{anchor=center, pos=0.125}, draw=none, from=1-1, to=2-2]
                        \end{tikzcd}
                    $$
                along with $M, N$ be $S$-modules\footnote{Which we ought to start seeing as objects of $\QCoh(S)$.} equipped with an $R$-linear differential operator:
                    $$D: M \to N$$
                of order $k \geq 0$. In addition, let $M'$ be an $S'$-module. Then the base-change:
                    $$D \tensor \id_{M'}: M \tensor_R M' \to N \tensor_R M'$$
                will be a $S$-linear differential operator, also of order $k$.
            \end{proposition}
                \begin{proof}
                    
                \end{proof}
    
        \subsubsection{Categories of D-modules on smooth varieties}
        
        \subsubsection{Singular support}
        
        \subsubsection{Holonomicity and duality}